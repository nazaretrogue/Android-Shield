\thispagestyle{empty}

\begin{center}
{\large\bfseries Android Shield \\ Detector de malware para Android }\\
\end{center}
\begin{center}
Nazaret Román Guerrero\\
\end{center}

%\vspace{0.7cm}

\vspace{0.5cm}
\noindent{\textbf{Palabras clave}: detector, malware, aplicación, permisos, análisis estático, SVM}
\vspace{0.7cm}

\noindent{\textbf{Resumen}\\

Desde que el sistema operativo de Android se lanzó al mercado en el año 2008, su uso ha crecido hasta alcanzar una cuota de mercado que superaba el 70\% en el enero de 2021. Su amplio uso, tanto en dispositivos móviles como en tabletas, relojes, televisores (\textit{SmartTV}) o, incluso automóviles, ha propiciado también el desarrollo de malware o software malicioso para este sistema.

Android contiene aplicaciones (no nativas) y funcionalidades (nativas) en las distintas capas del sistema operativo para mantener el dispositivo seguro. Mantener el dispositivo y las aplicaciones actualizadas y acceder a sitios seguros son medidas básicas de seguridad, como también lo es la gestión de permisos.

Android cuenta con un amplio conjunto de permisos que controlan el acceso a los distintos componentes hardware o software del dispositivo y las acciones que puede realizar cada aplicación. El sistema de permisos es eficaz para evitar ataques y proteger la privacidad del usuario siempre y cuando no se conceda acceso libre a aplicaciones que no deberían solicitar ciertos permisos considerados peligrosos.

En este trabajo nos centramos en analizar los permisos peligrosos solicitados por cada aplicación con la intención de determinar si es una aplicación inofensiva o, por el contrario, se trata de una aplicación maliciosa que se ha colado en nuestro sistema. Haciendo uso del conjunto de permisos que declara cada app, se ha desarrollado una aplicación para Android que detecte los permisos de cada aplicación y, según los permisos peligrosos solicitados, la clasifique en una aplicación benigna o maligna.

Para llevar a cabo esta actividad, se utiliza un modelo de \textit{Machine Learning} que hace uso de los algoritmos PCA y SVM. El PCA selecciona los permisos útiles para la clasificación y el SVM clasifica las aplicaciones según los permisos solicitados, centrándose solamente en aquellos permisos escogidos por el PCA. El modelo entrenado utiliza los permisos extraídos de distintas muestras de aplicaciones, tanto benignas como malignas, para después llevar a cabo la clasificación de nuevas aplicaciones. Con este modelo, el detector \textit{Android Shield} clasifica cada aplicación en benigna o maliciosa, alcanzando un ratio de acierto del 99\%.

\cleardoublepage

\begin{center}
	{\large\bfseries Android Shield: An Android malware detector}\\
\end{center}
\begin{center}
Nazaret Román Guerrero\\
\end{center}
\vspace{0.5cm}
\noindent{\textbf{Keywords}: detector, malware, application, permissions, static analysis, SVM}
\vspace{0.7cm}

\noindent{\textbf{Abstract}\\

Since the Android OS was launched on the market in 2008, its use has grown to reach a market share that exceeded 70\% in January 2021. Its wide use, both on mobile devices and tablets, smartwatches, smart TVs or even cars, has also led to the development of malware or malicious software for this system.

Android contains non-native applications and native functionalities in the different layers of the OS to keep the device safe and sound. Keeping the device and applications up-to-date and accessing only secure sites are basic security measures, as it is managing permissions.

Android has a wide set of permissions that control access to the different hardware or software components of the device and the actions that each application can perform. The permission system is effective in preventing attacks and protecting the user privacy as long as free access is not granted to applications that should not request certain permissions considered dangerous.

In this work we focus on analyzing the dangerous permissions requested by each application with the intention of determining whether it is a harmless application or, on the contrary, it is a malicious application that has crept into our system. Using the set of permissions declared by each app, an Android application has been developed to detect the permissions of each application and, according to the dangerous permissions requested, the app classifies each app into a benign or a malicious one.

To carry out this activity, a \textit{Machine Learning} model is used that makes use of the PCA and SVM algorithms. The PCA selects the useful permissions for classification and the SVM classifies the applications according to the requested permissions, focusing only on those permissions chosen by the PCA. The trained model uses the permissions extracted from different samples of applications, both benign and malware, and then carries out the classification of new applications. With this model, \textit{Android Shield} classifies each application as benign or malicious, reaching a success rate of 99\%.

\cleardoublepage

\chapter*{}
\thispagestyle{empty}

\noindent\rule[-1ex]{\textwidth}{2pt}\\[4.5ex]

D. \textbf{José Antonio Gómez Hernández}, Profesor del Departamento de Lenguajes y Sistemas Informáticos de la Universidad de Granada.

\vspace{0.5cm}

\textbf{Informa:}

\vspace{0.5cm}

Que el presente trabajo, titulado \textit{\textbf{Android Shield, Detector de malware para Android}},
ha sido realizado bajo su supervisión por \textbf{Nazaret Román Guerrero}, y autorizamos la defensa de dicho trabajo ante el tribunal que corresponda.

\vspace{0.5cm}

Y para que conste, expiden y firman el presente informe en Granada a 8 de julio de 2021 .

\vspace{1cm}

\textbf{El director:}

\vspace{5cm}

\noindent \textbf{José Antonio Gómez Hernández}

%\chapter*{Agradecimientos}
%\thispagestyle{empty}
%\vspace{1cm}