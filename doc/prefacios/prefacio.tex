\thispagestyle{empty}

\begin{center}
{\large\bfseries Android Shield \\ Detector de malware para Android }\\
\end{center}
\begin{center}
Nazaret Román Guerrero\\
\end{center}

%\vspace{0.7cm}

\vspace{0.5cm}
\noindent{\textbf{Palabras clave}: detector, malware, aplicación, permiso}
\vspace{0.7cm}

\noindent{\textbf{Resumen}\\

Desde que el sistema operativo de Android se lanzó al mercado en el año 2008, su uso ha crecido hasta alcanzar una cuota de mercado que superaba el 90\% en el año 2018. Su amplio uso, tanto en dispositivos móviles como en tabletas, relojes inteligentes, televisores inteligentes (\textit{SmartTV}) o, incluso, automóviles, ha propiciado también el desarrollo de malware o software malicioso.

Android cuenta con un amplio conjunto de aplicaciones (no nativas) y funcionalidades (nativas) en las distintas capas del sistema operativo para mantener el dispositivo seguro y a salvo de posibles ataques. Las aplicaciones intaladas, por lo general de una fuente segura y de confianza como la \textit{Play Store}, pasan un control de seguridad (a través de una funcionalidad de Google llamada \textit{Play Protect} que analiza la aplicación en busca de software maligno). Una vez instaladas en el dispositivo, todo depende del usuario.

Mantener el dispositivo y las aplicaciones actualizadas y acceder solo a sitios seguros son medidas básicas de seguridad, como también lo es la gestión de permisos. Android cuenta con un amplio conjunto de permisos que controlan el acceso a las distintas partes del dispositivo y las acciones que puede realizar cada aplicación. El sistema de permisos es eficaz para evitar ataques siempre y cuando el usuario no conceda acceso libre a aplicaciones que no deberían solicitar ciertos permisos considerados \underline{peligrosos}.

Siguiendo esta línea de acción, se ha desarrollado una aplicación para Android que detecte los permisos de cada aplicación y, según los permisos peligrosos solicitados, la clasifique en una aplicación benigna o maligna. Para llevar a cabo esta actividad, se utiliza un modelo entrenado mediante los permisos extraídos de distintas muestras de aplicaciones, tanto benignas como malignas, alcanzando un ratio de acierto del 99\%.
	

\cleardoublepage

\begin{center}
	{\large\bfseries Android Shield: An Android malware detector}\\
\end{center}
\begin{center}
Nazaret Román Guerrero\\
\end{center}
\vspace{0.5cm}
\noindent{\textbf{Keywords}: detector, malware, application, permission}
\vspace{0.7cm}

\noindent{\textbf{Abstract}\\

Since the Android OS was launched on the market in 2008, its use has grown to reach a market share that exceeded 90\% in 2018. Its wide use, both in mobile devices and tablets, smartwatches, smart TVs or even cars, has also led to the development of malware or malicious software.

Android has a wide set of non-native applications and native functionalities in the different layers of the OS to keep the device safe and sound from possible attacks. Installed apps, usually from a trusted source like the \textit{Play Store}, pass a security check (through a Google feature called \textit{Play Protect} that scans the app for malicious software). Once the software is installed on the device, it all depends on the user.

Keeping the device and applications up-to-date and accessing only secure sites are basic security measures, as it is managing permissions. Android has a wide set of permissions that control the access to the different parts of the hardware, and also the actions that each application can perform. The permission system is effective in preventing attacks as long as the user does not grant free access to applications that should not request certain permissions considered \underline{dangerous}.

Following this line of action, we have developed an Android application that detects the permissions of each application and, according to the dangerous permissions requested, classifies it into a benign or a malicious application. To carry out this activity, a model trained by sets of permissions extracted from different samples of applications, both benign apps and malware, has been used, reaching a success rate of 99\%.

\cleardoublepage

\chapter*{}
\thispagestyle{empty}

\noindent\rule[-1ex]{\textwidth}{2pt}\\[4.5ex]

D. \textbf{José Antonio Gómez Hernández}, Profesor del Área de XXXX del Departamento de Lenguajes y Sistemas Informáticos de la Universidad de Granada.

\vspace{0.5cm}

\textbf{Informa:}

\vspace{0.5cm}

Que el presente trabajo, titulado \textit{\textbf{Android Shield, Detector de malware para Android}},
ha sido realizado bajo su supervisión por \textbf{Nazaret Román Guerrero}, y autorizamos la defensa de dicho trabajo ante el tribunal que corresponda.

\vspace{0.5cm}

Y para que conste, expiden y firman el presente informe en Granada a X de junio de 2021 .

\vspace{1cm}

\textbf{El director:}

\vspace{5cm}

\noindent \textbf{José Antonio Gómez Hernández}

\chapter*{Agradecimientos}
\thispagestyle{empty}

       \vspace{1cm}