\chapter{Anexos}

\section{Anexo I: Lista de permisos peligrosos}
\label{lista_permisos}

Los permisos peligrosos son aquellos permisos que, en tiempo de ejecución, dan acceso a ciertas funcionalidades, partes del sistema o datos restringidos (o sensibles) a la aplicación. Pueden afectar significativamente al funcionamiento de otras apps o incluso del sistema en general, así que se debe tener cuidado cuando una aplicación solicita algun permiso considerado peligroso.

El listado de todos los permisos peligrosos se ha extraído de la documentación oficial de Android (aquellos etiquetados como \textit{Protection level: dangerous} tras la actualización al nivel 30 de la API) y puede verse en \hypersetup{citecolor=red}\cite{riskyperm}.

A pesar de que puede haber otros permisos (como el acceso a Internet) que puedan ser considerados peligrosos, no son determinantes a la hora de diferenciar entre aplicaciones benignas y malware, pues la gran mayoría de aplicaciones requieren permiso a Internet, mientras que otros no considerados peligrosos sí pueden ayudar en la clasificación, como se demostró en\hypersetup{citecolor=red}\cite{sigpid}.

En este trabajo se ha considerado utilizar solamente los permisos peligrosos porque son los que exponen más al usuario, ya que son aquellos que solicitan acceso a cámara, micrófono, datos de localización, almacenamiento, contactos, acceso a los sensores corporales (en el caso de tener accesorios como un \textit{fitness track} que mida el pulso)... Por tanto, los permisos utilizados son:

\begin{itemize}
	\item ACCEPT\_HANDOVER: permite que la aplicación de llamadas continúe con una llamada que se inició en otra aplicación.
	\item ACCESS\_BACKGROUND\_LOCATION: permite acceder a la localización estando en segundo plano.
	\item ACCESS\_COARSE\_LOCATION: permite acceder a la localización aproximada.
	\item ACCESS\_FINE\_LOCATION: permite acceder a la localización con precisión.
	\item ACCESS\_MEDIA\_LOCATION: permite acceder a la localización geográfica en sus archivos. Por ejemplo, si un usuario hace una fotografía en un monumento, la imagen contendrá la localización en la que se tomó y una aplicación podrá acceder a dicha información.
	\item ACTIVITY\_RECOGNITION: permite acceder a la actividad física (relacionado con \textit{fitness track}).
	\item ADD\_VOICEMAIL: permite que una aplicación cree mensajes de voz para el contestador.
	\item ANSWER\_PHONE\_CALLS: permite que una aplicación responda llamadas entrantes.
	\item BODY\_SENSORS: permite acceder a la información recopilada por ACTIVITY\_RECOGNITION.
	\item CALL\_PHONE: permite a una aplicación hacer llamadas sin necesidad de acceder al marcador de la aplicación del teléfono.
	\item CAMERA: permite acceder a la cámara.
	\item GET\_ACCOUNTS: permite acceder al listado de aplicaciones guardadas en la aplicación de gestión de cuentas del dispositivo.
	\item PROCESS\_OUTGOING\_CALLS: aunque este permiso quedó obsoleto en la API 29, se ha incluido porque muchas aplicaciones anteriores a la última actualización de Android lo utilizan. Permite a una aplicación ver el número marcado con el objetivo de poder redirigir la llamada a otro teléfono o para colgar. 
	\item READ\_CALENDAR: permite leer la información almacenada en el calendario del usuario.
	\item READ\_CALL\_LOG: permite leer el historial de llamadas del dispositivo.
	\item READ\_CONTACTS: permite leer los datos de los contactos almacenados en la agenda, como nombres, direcciones de correo o ubicación. No permite leer los números de teléfono.
	\item READ\_EXTERNAL\_STORAGE: permite leer los datos almacenados en el almacenamiento externo.
	\item READ\_PHONE\_NUMBERS: permite leer los números de teléfono de los contactos.
	\item READ\_PHONE\_STATE: permite la lectura del estado del teléfono, como la información de red, las llamadas salientes y cualquier cuenta almacenada en el dispositivo.
	\item READ\_SMS: permite leer los mensajes SMS del dispositivo.
	\item RECEIVE\_MMS: permite monitorizar la entrada de mensajes MMS en el dispositivo a través de una aplicación diferente a la de mensajería.
	\item RECEIVE\_SMS: permite monitorizar la entrada de mensajes SMS en el dispositivo a través de una aplicación diferente a la de mensajería.
	\item RECEIVE\_WAP\_PUSH: permite a una aplicación recibir mensajes WAP Push, un tipo de SMS que se utiliza para acceder a una página WAP sin necesidad de introducir la URL de la página en el navegador del teléfono.
	\item RECORD\_AUDIO: permite a la aplicación acceder al micrófono, grabar y almacenar audios.
	\item SEND\_SMS: permite a una aplicación escribir y enviar mensajes SMS.
	\item USE\_SIP: permite a una aplicación hacer uso del servicio SIP para hacer llamadas.
	\item WRITE\_CALENDAR: permite modificar, eliminar o crear nueva información en el calendario del usuario.
	\item WRITE\_CALL\_LOG: permite escribir (pero no leer) en el historial de llamadas.
	\item WRITE\_CONTACTS: permite a la aplicación escribir nuevos datos, modificar o eliminar contactos en el dispositivo.
	\item WRITE\_EXTERNAL\_STORAGE: permite escribir nuevos datos, modificar o eliminar datos almacenados en el almacenamiento externo. De manera implícita, al activar este permiso también se activa el permiso de READ\_EXTERNAL\_STORAGE.
\end{itemize}