\chapter{Conclusiones y trabajos futuros}

Para terminar, hagamos una pequeña reflexión con todo lo visto, tanto en lo referente a los antecedentes que nos han llevado a desarrollar este trabajo como a trabajo en sí y lo que éste implica.

Los ataques a dispostivos Android son una de las principales amenazas a las que nos enfrentamos a diario. Puesto que nuestros teléfonos móviles son una de las tecnologías más extendidas a lo largo del mundo y una de las más utilizadas cada día, gran parte de nuestra vida está contenida en ellos, lo que nos hace vulnerables y un objetivo atractivo para los atacantes. Nadie está a salvo, aunque creamos que al ser usuarios ``comunes'' no estamos expuestos al mismo peligro que las grandes empresas. Tal vez el riesgo que corremos es menor, pero no es nulo. Es importante mantener el dispositivo actualizado, instalar solo aplicaciones de fuentes confiables y dar y revocar solo los permisos estrictamente necesarios para evitar abrir brechas de seguridad por las que puedan colarse los atacantes.

Como se ha explicado a lo largo de esta memoria, hay diferentes formas de enfocar el problema para evitar lo máximo posible los riesgos a los que nos encontramos expuestos, pero ninguna de estas soluciones es 100\% eficaz. Esto significa que siempre, por muy cuidadosos que seamos, habrá una posibilidad de que nuestro dispositivo sea atacado y nuestros datos, puestos en riesgo. La aproximación que ofrece la aplicación desarrollada en este proyecto es una solución adecuada para la detección de malware en Android sin reducir en exceso el rendimiento del dispositivo, lo que permite que el usuario pueda tomar medidas para eliminar el malware de su teléfono.

A pesar de los errores que pueda cometer en la clasificación, la aplicación de \textit{Android Shield} es una aproximación válida para controlar el dispositivo y saber si ha sido infectado sin que el rendimiento se vea casi afectado. El análisis llevado a cabo muestra una precisión del 99\%, lo que significa que esta aplicación es capaz de detectar la mayor parte del malware mejor que otros trabajos similares, aunque sigue habiendo falsos negativos que pueden ser reducidos para mejorar aún más la precisión.

Una ventaja clara que posee la aplicación es que utiliza un análisis estático, lo que significa que no requiere la engorrosa tarea de ejecutar previamente en un entorno aislado la aplicación sospechosa de ser malware. Además, al llevarse a cabo el análisis dentro del dispositivo, la ejecución es más rápida.

El objetivo principal para mejorar la aplicación es conseguir que detecte malware del día cero. Se podría continuar desarrollando la aplicación para mejorar el análisis con más experimentación y ajustando los parámetros del algoritmo, de manera que también fuera capaz de detectar malware que nunca había aparecido antes y que cuenta con una combinación de permisos declarados que el modelo entrenado no ha visto.

Para mejorar más \textit{Android Shield}, se podría hacer más atractiva la interfaz de usuario y añadir nuevas opciones de análisis, como por ejemplo, en el caso de que un usuario instale una nueva aplicación y quiera analizarla pero ya haya analizado el resto de apps de su dispositivo, se podría añadir una opción en la que se elija qué aplicaciones se quieren analizar en lugar de analizarlas todas cada vez que ejecutamos \textit{Android Shield}. Una tercera mejora que se podría incluir es una opción en la que al pulsar sobre una aplicación, se mostrara un listado con los permisos activos (los que están en uso) de dicha aplicación.

Un avance muy intersante sería conseguir que \textit{Android Shield} clasificara la familia a la que pertenece el malware que haya detectado. Esa mejora requeriría un análisis más en profundidad de cada aplicación para ser capaz de determinar el tipo de malware al que nos enfrentamos.

La ciberseguridad en Android es un campo en el que queda mucho estudio por delante, pero poco a poco nos vamos acercando a una defensa eficaz. Un \textit{escudo} para Android que nos permita detectar la mayoría de las amenazas a las que, sin ser conscientes, nos vemos abocados.